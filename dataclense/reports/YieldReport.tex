%% http://biostat.mc.vanderbilt.edu/wiki/Main/SweaveTemplate#Using_SweaveListingUtils_http_bi

\documentclass[10pt,a4paper]{article}\usepackage{graphicx, color}
%% maxwidth is the original width if it is less than linewidth
%% otherwise use linewidth (to make sure the graphics do not exceed the margin)
\makeatletter
\def\maxwidth{ %
  \ifdim\Gin@nat@width>\linewidth
    \linewidth
  \else
    \Gin@nat@width
  \fi
}
\makeatother

\IfFileExists{upquote.sty}{\usepackage{upquote}}{}
\definecolor{fgcolor}{rgb}{0.2, 0.2, 0.2}
\newcommand{\hlnumber}[1]{\textcolor[rgb]{0,0,0}{#1}}%
\newcommand{\hlfunctioncall}[1]{\textcolor[rgb]{0.501960784313725,0,0.329411764705882}{\textbf{#1}}}%
\newcommand{\hlstring}[1]{\textcolor[rgb]{0.6,0.6,1}{#1}}%
\newcommand{\hlkeyword}[1]{\textcolor[rgb]{0,0,0}{\textbf{#1}}}%
\newcommand{\hlargument}[1]{\textcolor[rgb]{0.690196078431373,0.250980392156863,0.0196078431372549}{#1}}%
\newcommand{\hlcomment}[1]{\textcolor[rgb]{0.180392156862745,0.6,0.341176470588235}{#1}}%
\newcommand{\hlroxygencomment}[1]{\textcolor[rgb]{0.43921568627451,0.47843137254902,0.701960784313725}{#1}}%
\newcommand{\hlformalargs}[1]{\textcolor[rgb]{0.690196078431373,0.250980392156863,0.0196078431372549}{#1}}%
\newcommand{\hleqformalargs}[1]{\textcolor[rgb]{0.690196078431373,0.250980392156863,0.0196078431372549}{#1}}%
\newcommand{\hlassignement}[1]{\textcolor[rgb]{0,0,0}{\textbf{#1}}}%
\newcommand{\hlpackage}[1]{\textcolor[rgb]{0.588235294117647,0.709803921568627,0.145098039215686}{#1}}%
\newcommand{\hlslot}[1]{\textit{#1}}%
\newcommand{\hlsymbol}[1]{\textcolor[rgb]{0,0,0}{#1}}%
\newcommand{\hlprompt}[1]{\textcolor[rgb]{0.2,0.2,0.2}{#1}}%

\usepackage{framed}
\makeatletter
\newenvironment{kframe}{%
 \def\at@end@of@kframe{}%
 \ifinner\ifhmode%
  \def\at@end@of@kframe{\end{minipage}}%
  \begin{minipage}{\columnwidth}%
 \fi\fi%
 \def\FrameCommand##1{\hskip\@totalleftmargin \hskip-\fboxsep
 \colorbox{shadecolor}{##1}\hskip-\fboxsep
     % There is no \\@totalrightmargin, so:
     \hskip-\linewidth \hskip-\@totalleftmargin \hskip\columnwidth}%
 \MakeFramed {\advance\hsize-\width
   \@totalleftmargin\z@ \linewidth\hsize
   \@setminipage}}%
 {\par\unskip\endMakeFramed%
 \at@end@of@kframe}
\makeatother

\definecolor{shadecolor}{rgb}{.97, .97, .97}
\definecolor{messagecolor}{rgb}{0, 0, 0}
\definecolor{warningcolor}{rgb}{1, 0, 1}
\definecolor{errorcolor}{rgb}{1, 0, 0}
\newenvironment{knitrout}{}{} % an empty environment to be redefined in TeX

\usepackage{alltt}
\usepackage[OT1]{fontenc}
\usepackage{fullpage}
\usepackage{amsmath}



\title{Deployment Data Cleanup}
\author{D Goldsmith, R Wilkins}

%Hide Messages from loading R packages
%opts_chunk$set(message=FALSE}. 

%Wrap the output of listings 

%\usepackage{listings}



\begin{document}
\maketitle

\section{Introduction}

Below is a description of the data summary process, how the information is
processed an yields calculated.

It takes the form of a walk though of the yield calculation process, and
includes code snippets from the R script used to generate the data.

The document was generated in R and Latex, using the Knitr package


\section{Summary Table}
A new table added to the database, the summary table is intended to hold summary
statistics on deployments.  This means that future work can avoid having to
process entire data sets when dealing with yield, or other summarised functions.  

The summary table takes the same form as the reading table, with an additional
\emph{summary type} column,  these summary types are taken from a lookup
table in the database.

Database Rows, and expected inputs are given below

\begin{table}[htbp]
  \centering
  \begin{tabular}{l l p{6cm}}
    Row & Type & Description \\ \hline
    Time & PK,Required & Timestamp of summary, In general I would expect
  this to use midnight to summarise a complete day. However, if more detailed
  summaries (such as hourly) are needed, this should not be a problem.\\
    nodeId &PK,Required & Id of node that this summary is from\\
    sensorTypeId &PK & Id of sensor that this summary is from,  this
  can be left NULL to indicate whole node summary samples (for example yield)\\
    summaryTypeId &FK &  Id of summary type.\\
    locationId & FK & Id of location this node is from,  to keep
  parity with the reading table\\
    value & float& Value of the summary\\
    textValue & string(30)& Optional text description of the summary, for example ``Hot''
  if we are dealing with exposure graphs.\\
    
  \end{tabular}
  \caption{Summary Table Description}
\end{table}

\section{Scripts}
This section has a description of the scripts used process the data, and combine
all samples into one database.

These scripts are designed to work with the new format (location aware) database format.

They can found in the \emph{dataclense} directory of the \emph{cogent-house/djgoldsmith-devel} repository.

\begin{description}
    \item[processCC.py]  Transfers current cost data from the old style sqlite
      database, into the new format database.
    \item[processAr.py] Transfers data from an Archrock postgresql database into the
      new format database.
    \item[getStats.R] R script that calculates yields for each deployment in a given database
    \item[calcKwh.R] R script to caluclate KwH usage from current cost readings.
\end{description}

Further details of these scripts are given below

\section{getStats.R}




This script calculates summary statistics for all houses in a given database.
The statistics are output in two formats.
\begin{itemize}
  \item \emph{.csv} file with summary output for this database
  \item update rows in the \emph{summary} table given these statistics
\end{itemize}

To run the script modify the source file with the relevant database access
name. Then run the script through R. 


\subsection{Script initialisation}

\begin{itemize}
\item Load the relevant R librarys
\item Connects to the database
\item Loads the Relevant Lookup tables into memory.
  \begin{itemize}
  \item Houses Table
  \item Sensor Table (For Calibration)
  \item Sensor Type Table
  \item Summary Type Table
  \end{itemize}
\end{itemize}

\begin{knitrout}
\definecolor{shadecolor}{rgb}{0.969, 0.969, 0.969}\color{fgcolor}\begin{kframe}
\begin{alltt}
\hlcomment{# Load Relevant Libraries}
\hlfunctioncall{library}(RMySQL)
\end{alltt}


{\ttfamily\noindent\itshape\textcolor{messagecolor}{\#\# Loading required package: methods}}

{\ttfamily\noindent\itshape\textcolor{messagecolor}{\#\# Loading required package: DBI}}\begin{alltt}
\hlfunctioncall{library}(ggplot2)
\hlfunctioncall{library}(plyr)

\hlcomment{# Setup Database Connection}
drv <- \hlfunctioncall{dbDriver}(\hlstring{"MySQL"})
\hlcomment{# con <- dbConnect(drv,dbname='mainStore',user='chuser')}
con <- \hlfunctioncall{dbConnect}(drv, dbname = \hlstring{"mainStore"}, user = \hlstring{"root"}, password = \hlstring{"Ex3lS4ga"})

\hlcomment{# Load the Relevant lookup tables into memeory}
allHouses <- \hlfunctioncall{dbGetQuery}(con, statement = \hlstring{"SELECT * FROM House WHERE address != \hlstring{'ERROR-DATA'}"})
summaryData <- \hlfunctioncall{dbReadTable}(con, \hlstring{"SummaryType"})
calibrationData <- \hlfunctioncall{dbReadTable}(con, \hlstring{"Sensor"})
sensorType <- \hlfunctioncall{dbReadTable}(con, \hlstring{"SensorType"})

\hlcomment{## Sensors we are interested in (For Yield Calculateions)}
sensorTypeList <- \hlfunctioncall{subset}(sensorType, name == \hlstring{"Temperature"} | name == \hlstring{"Humidity"} | 
    name == \hlstring{"Light PAR"} | name == \hlstring{"Light TSR"} | name == \hlstring{"CO2"} | name == \hlstring{"Air Quality"} | 
    name == \hlstring{"VOC"} | name == \hlstring{"Battery Voltage"} | name == \hlstring{"Power"})

\hlcomment{# Create a temporary table to hold summary informtion}
houseData <- \hlfunctioncall{data.frame}(address = allHouses$address, dbStart = NA, dbEnd = NA, 
    dataStart = NA, dataEnd = NA, totalNodes = NA, coNodes = NA, yield = NA, 
    yieldSD = NA, yieldMin = NA, yieldMax = NA, totalSamples = NA, yieldDays = NA)

\hlcomment{# Choose a particular house (for the demo)}
i = 16
THEHOUSE <- allHouses[i, ]
hseName <- THEHOUSE$address
\end{alltt}
\end{kframe}
\end{knitrout}


At the end of this we have 
\begin{enumerate}
\item A connection to the Database
\item  A collection of lookup tables used later in the application
\item  One main dataframe, to hold the summary information generated during the
summarising process.
\item  Selected a house, in this case 131Jodrell
\end{enumerate}

\subsection{Loading the Inital Dataset}

This section of code fetches the relevant data for each property.

First the relevant house is fetched, and the start and end dates processed to
convert to a format R is comfortable with.

Next the Relevant locations are fetched using the following query
This query also fetches the corresponding room names for each location for use
in a lookup table if required.

\begin{knitrout}
\definecolor{shadecolor}{rgb}{0.969, 0.969, 0.969}\color{fgcolor}\begin{kframe}
\begin{alltt}
\hlcomment{# As the function parameter is a row number, Work out which House we are}
\hlcomment{# dealing with. print(paste('Processing House ',hseName))}
rowNo <- \hlfunctioncall{which}(houseData$address == hseName)
cleanName <- \hlfunctioncall{gsub}(\hlstring{" "}, \hlstring{""}, hseName)

\hlcomment{# Get House from the database}
houseQry <- \hlfunctioncall{paste}(\hlstring{"SELECT * FROM House WHERE address = \hlstring{'"}, hseName, \hlstring{"'}"}, sep = \hlstring{""})
theHouse <- \hlfunctioncall{dbGetQuery}(con, statement = houseQry)

\hlcomment{# Process start and end dates into a format that R is happy with}
theHouse$sd <- \hlfunctioncall{tryCatch}(\{
    \hlfunctioncall{as.POSIXlt}(theHouse$startDate, tz = \hlstring{"GMT"})
\}, error = \hlfunctioncall{function}(e) \{
    NA
\})

theHouse$ed <- \hlfunctioncall{tryCatch}(\{
    \hlfunctioncall{as.POSIXlt}(theHouse$endDate, tz = \hlstring{"GMT"})
\}, error = \hlfunctioncall{function}(e) \{
    NA
\})

\hlcomment{# Fetch Locations from the database}
locQry <- \hlfunctioncall{paste}(\hlstring{"SELECT * FROM Location as Loc "}, \hlstring{" LEFT OUTER JOIN Room as Room "}, 
    \hlstring{" ON Loc.roomId = Room.id "}, \hlstring{" WHERE houseId ="}, theHouse$id, \hlstring{" AND Room.name NOT LIKE \hlstring{'PhyNet%'} "}, 
    \hlstring{" AND Room.name NOT \hlfunctioncall{IN} (\hlstring{'Hot Water'},\hlstring{'Cold Water'},\hlstring{'Flow'},\hlstring{'Return'},\hlstring{'Hot'},\hlstring{'Cold'},\hlstring{'HotWater'},\hlstring{'ColdWater'}) "}, 
    sep = \hlstring{""})
locations <- \hlfunctioncall{dbGetQuery}(con, statement = locQry)

\hlcomment{# Convert into a string so we can put the data into the next query}
locationIds <- \hlfunctioncall{paste}(locations$id, collapse = \hlstring{","})
\end{alltt}
\end{kframe}
\end{knitrout}


Next the relevant dataset is fetched. 
Rather than fetch the entire dataset, and process it on the local machine, the
query used makes use of SQL summarising functions to group the data by day. 
The query fetches the following values:

\begin{description}
\item[nodeId] Used to Identify the Node
\item[locationId] Used to Identify the Location
\item[date] Timestamp converted to a date (i.e. with the time removed)
\item[minTime] Time of first sample on this date
\item[maxTime] Time of last sample on this date
\item[minVal] Minimum reported Value on this date
\item[maxVal] Maximum reported value on this date
\item[meanVal] Average value reported on this date
\end{description}

 
\begin{knitrout}
\definecolor{shadecolor}{rgb}{0.969, 0.969, 0.969}\color{fgcolor}\begin{kframe}
\begin{alltt}
houseData[rowNo, ]$dbStart <- theHouse$startDate
houseData[rowNo, ]$dbEnd <- theHouse$endDate

\hlcomment{# Setup the Query}
sTypes <- \hlfunctioncall{paste}(sensorTypeList$id, collapse = \hlstring{","})
dataQry <- \hlfunctioncall{paste}(\hlstring{"SELECT nodeId, type, locationId, \hlfunctioncall{DATE}(time) as date,\hlfunctioncall{count}(*) as count,"}, 
    \hlstring{" \hlfunctioncall{min}(time) as minTime,\hlfunctioncall{max}(time) as maxTime,"}, \hlstring{" \hlfunctioncall{min}(value) as minVal, \hlfunctioncall{max}(value) as maxVal, \hlfunctioncall{avg}(value) as meanVal"}, 
    \hlstring{" FROM Reading WHERE locationId \hlfunctioncall{IN} ("}, locationIds, \hlstring{") "}, \hlstring{" AND type \hlfunctioncall{IN} ("}, 
    sTypes, \hlstring{") "}, \hlstring{" GROUP BY nodeId,type,\hlfunctioncall{DATE}(time)"}, sep = \hlstring{""})

\hlcomment{# Fetch the data}
houseSummary <- \hlfunctioncall{dbGetQuery}(con, statement = dataQry)
\hlcomment{# Add a time object so that makes sense to R}
houseSummary$dt <- \hlfunctioncall{as.POSIXlt}(houseSummary$date, tz = \hlstring{"GMT"})
houseSummary$DT <- \hlfunctioncall{as.Date}(houseSummary$dt)
\hlcomment{# And add some extra columns to hold summary information on bad samples}
houseSummary$countWithBad <- houseSummary$count
\end{alltt}
\end{kframe}
\end{knitrout}


\subsection{Calibration}
For the purpouses of error detection, it should be sufficent to examine the
minimum and maximum values.

The dataset is calibrated by merging with the calibration table, to append the
calibration slope and offset (where known), where there is no calibration data,
a default value of 1 for slope, and 0 for offset are used.  
From this a calibrated version of the Min,Max and Mean values is calculated.

\begin{knitrout}
\definecolor{shadecolor}{rgb}{0.969, 0.969, 0.969}\color{fgcolor}\begin{kframe}
\begin{alltt}
\hlcomment{# Merge with Calibration Data}
tmp <- \hlfunctioncall{merge}(houseSummary, calibrationData, by.x = \hlfunctioncall{c}(\hlstring{"nodeId"}, \hlstring{"type"}), by.y = \hlfunctioncall{c}(\hlstring{"nodeId"}, 
    \hlstring{"sensorTypeId"}), all.x = TRUE)

\hlcomment{# Add a dafualt calibration for nodes that have none}
noCalibIdx <- \hlfunctioncall{which}(\hlfunctioncall{is.na}(tmp$id) == TRUE)
tmp[noCalibIdx, ]$calibrationSlope <- 1
tmp[noCalibIdx, ]$calibrationOffset <- 0

\hlcomment{# Calibrate}
tmp$calibMin <- (tmp$minVal * tmp$calibrationSlope) + tmp$calibrationOffset
tmp$calibMax <- (tmp$maxVal * tmp$calibrationSlope) + tmp$calibrationOffset
tmp$calibMean <- (tmp$meanVal * tmp$calibrationSlope) + tmp$calibrationOffset
\hlcomment{# Remove Data outside of given bands}
tmp$badValue <- NA
\end{alltt}
\end{kframe}
\end{knitrout}


Figure~\ref{fig:calibData} shows calibrated temperature data, before any error handling
takes place.

% 

\begin{figure}[ht]
\begin{knitrout}
\definecolor{shadecolor}{rgb}{0.969, 0.969, 0.969}\color{fgcolor}\begin{kframe}
\begin{alltt}
plt <- \hlfunctioncall{ggplot}(\hlfunctioncall{subset}(tmp, type == 0))
plt <- plt + \hlfunctioncall{geom_point}(\hlfunctioncall{aes}(dt, calibMean, color = \hlstring{"calibrated"}))
plt <- plt + \hlfunctioncall{geom_errorbar}(\hlfunctioncall{aes}(dt, ymin = calibMin, ymax = calibMax, color = \hlstring{"calibrated"}))
plt <- plt + \hlfunctioncall{geom_point}(\hlfunctioncall{aes}(dt, meanVal, color = \hlstring{"uncalibrated"}))
plt <- plt + \hlfunctioncall{opts}(title = \hlfunctioncall{paste}(hseName, \hlstring{"Summary Temperature \hlfunctioncall{Data} (calibrated)"}))
plt <- plt + \hlfunctioncall{xlab}(\hlstring{"Date"}) + \hlfunctioncall{ylab}(\hlstring{"Value"})
plt <- plt + \hlfunctioncall{facet_grid}(nodeId ~ .)
plt
\end{alltt}
\end{kframe}
\includegraphics[width=\maxwidth]{figure/unnamed-chunk-7} 

\end{knitrout}

\caption{Calibrated Temperature Data}
\label{fig:calibData}
\end{figure}

\subsection{Error Detection}
Errors in the data are detected using a basic threshold based approach: Dates
where the minimum or maximum values fall out of the ranges in
Table~\ref{tab:accepted-Range} are marked as invalid.

\emph{NOTE:} Given that with the Telos the data for temperature and humidity
sensors tends to go bad when the battery voltage falls below 2.3V,  it may be
appropriate to filter any temperature values when the battery drops below this level.

\begin{table}[htbp]
\begin{tabular}{l l l}
  Parameter & Minimum Threshold & Maximum Threshold \\
  \hline
  Temperature & -10 & 50 \\
  Humidity & 0 & 100 \\
  Co2 & 0 & 6000
\end{tabular}
\caption{Accepted Ranges}
\label{tab:accepted-Range}
\end{table}

\begin{knitrout}
\definecolor{shadecolor}{rgb}{0.969, 0.969, 0.969}\color{fgcolor}\begin{kframe}
\begin{alltt}
\hlcomment{# Set a flag for data outside of certain bands}
tmp$badValue <- NA

\hlcomment{# Temperatre}
badRows <- \hlfunctioncall{which}(tmp$type == 0 & (tmp$calibMin < -10 | tmp$calibMax > 50))
\hlfunctioncall{if} (\hlfunctioncall{length}(badRows) > 0) \{
    tmp[badRows, ]$badValue <- TRUE
\}
\hlcomment{# Humidity}
badRows <- \hlfunctioncall{which}(tmp$type == 2 & (tmp$calibMin < 0 | tmp$calibMax > 100))
\hlfunctioncall{if} (\hlfunctioncall{length}(badRows) > 0) \{
    tmp[badRows, ]$badValue <- TRUE
\}
\hlcomment{# Co2}
badRows <- \hlfunctioncall{which}(tmp$type == 8 & (tmp$calibMin < 0 | tmp$calibMax > 6000))
\hlfunctioncall{if} (\hlfunctioncall{length}(badRows) > 0) \{
    tmp[badRows, ]$badValue <- TRUE
\}
\hlcomment{# TODO Rather than threshold by anything else, Threshold Electricity by}
\hlcomment{# something sensible}
badRows <- \hlfunctioncall{which}(tmp$type == 6 & (tmp$calibMin < 0 | tmp$calibMax > 5))
\hlfunctioncall{if} (\hlfunctioncall{length}(badRows) > 0) \{
    tmp[badRows, ]$badValue <- TRUE
\}
\end{alltt}
\end{kframe}
\end{knitrout}


Where data is marked as invalid, the full data stream for that day and node is
collected.

This full data stream is then calibrated and the bad readings removed using the
method above.

Each day is then summarised, and the original entry in the \emph{houseSummary}
data frame replaced with the new summary values.  Additionally, a new column
\emph{badCount} is added to the data table to represent the total number of
samples before any data is removed.  This may allow some insight into the number
of bad samples per deployment.

%I would love to break this down, but I cant work out how to split blocks of code
%to keep the IF statement together.

\begin{knitrout}
\definecolor{shadecolor}{rgb}{0.969, 0.969, 0.969}\color{fgcolor}\begin{kframe}
\begin{alltt}
\hlcomment{# Work out the correct SQL statement}
sqlValues <- \hlfunctioncall{subset}(tmp, badValue == TRUE)

\hlfunctioncall{if} (\hlfunctioncall{nrow}(sqlValues) > 0) \{
    uniqueNode <- \hlfunctioncall{paste}(\hlfunctioncall{unique}(sqlValues$nodeId), collapse = \hlstring{","})
    uniqueDate <- \hlfunctioncall{paste}(\hlfunctioncall{shQuote}(\hlfunctioncall{unique}(sqlValues$dt)), collapse = \hlstring{","})
    theQry = \hlfunctioncall{paste}(\hlstring{"SELECT * FROM Reading WHERE"}, \hlstring{" NodeId \hlfunctioncall{IN} ("}, uniqueNode, 
        \hlstring{")"}, \hlstring{" AND \hlfunctioncall{Date}(time) \hlfunctioncall{IN} ("}, uniqueDate, \hlstring{")"}, \hlstring{" AND type \hlfunctioncall{IN} ("}, \hlfunctioncall{paste}(sensorTypeList$id, 
            collapse = \hlstring{","}), \hlstring{")"}, sep = \hlstring{""})
    
\hlcomment{    # Fetch all that data}
    fixData <- \hlfunctioncall{dbGetQuery}(con, statement = theQry)
    
\hlcomment{    # Merge with Calibration Stuff}
    fixCalib <- \hlfunctioncall{merge}(fixData, calibrationData, by.x = \hlfunctioncall{c}(\hlstring{"nodeId"}, \hlstring{"type"}), 
        by.y = \hlfunctioncall{c}(\hlstring{"nodeId"}, \hlstring{"sensorTypeId"}), all.x = TRUE)
    noCalibIdx <- \hlfunctioncall{which}(\hlfunctioncall{is.na}(fixCalib$id) == TRUE)
    rowcount <- \hlfunctioncall{length}(noCalibIdx)
    \hlfunctioncall{if} (rowcount > 0) \{
        fixCalib[noCalibIdx, ]$calibrationSlope <- 1
        fixCalib[noCalibIdx, ]$calibrationOffset <- 0
    \}
    
\hlcomment{    # Calibrate}
    fixCalib$calibValue <- (fixCalib$value * fixCalib$calibrationSlope) + fixCalib$calibrationOffset
    fixCalib$ts <- \hlfunctioncall{as.POSIXlt}(fixCalib$time, tz = \hlstring{"GMT"})
    fixCalib$dt <- \hlfunctioncall{as.Date}(fixCalib$ts)
    
\hlcomment{    # Remove all the bad data}
    fixCalib$badValue <- FALSE
    badRows <- \hlfunctioncall{which}(fixCalib$type == 0 & (fixCalib$calibValue < -10 | fixCalib$calibValue > 
        50))
    \hlfunctioncall{if} (\hlfunctioncall{length}(badRows) > 0) \{
        fixCalib[badRows, ]$badValue <- TRUE
        fixCalib[badRows, ]$value = NA
    \}
    badRows <- \hlfunctioncall{which}(fixCalib$type == 2 & (fixCalib$calibValue < 0 | fixCalib$calibValue > 
        100))
    \hlfunctioncall{if} (\hlfunctioncall{length}(badRows) > 0) \{
        fixCalib[badRows, ]$badValue <- TRUE
        fixCalib[badRows, ]$value = NA
    \}
    badRows <- \hlfunctioncall{which}(fixCalib$type == 8 & (fixCalib$calibValue < 0 | fixCalib$calibValue > 
        6000))
    \hlfunctioncall{if} (\hlfunctioncall{length}(badRows) > 0) \{
        fixCalib[badRows, ]$badValue <- TRUE
        fixCalib[badRows, ]$value = NA
    \}
\hlcomment{    # We could do with removing any temperture / humidity data where the}
\hlcomment{    # battery level is below XXX}
    badRows <- \hlfunctioncall{which}(fixCalib$type == 6 & (fixCalib$calibValue < 0 | fixCalib$calibValue > 
        5))
    \hlfunctioncall{if} (\hlfunctioncall{length}(badRows) > 0) \{
        fixCalib[badRows, ]$badValue <- TRUE
        fixCalib[badRows, ]$value = NA
    \}
    
\hlcomment{    # Summarise so its in the same format as the overall data}
    fixSummary <- \hlfunctioncall{ddply}(fixCalib, \hlfunctioncall{.}(nodeId, locationId, type, dt), summarise, 
        minVal = \hlfunctioncall{min}(value, na.rm = TRUE), maxVal = \hlfunctioncall{max}(value, na.rm = TRUE), 
        meanVal = \hlfunctioncall{mean}(value, na.rm = TRUE), minTime = \hlfunctioncall{min}(ts), maxTime = \hlfunctioncall{max}(ts), 
        count = \hlfunctioncall{length}(value), naCount = \hlfunctioncall{sum}(\hlfunctioncall{is.na}(value)), tCount = \hlfunctioncall{length}(value) - 
            \hlfunctioncall{sum}(\hlfunctioncall{is.na}(value)))
    
\hlcomment{    # Remove the Infs put in by sumary functions where there is no data.}
    infDates <- \hlfunctioncall{which}(\hlfunctioncall{is.infinite}(fixSummary$minVal) == TRUE)
    \hlfunctioncall{if} (\hlfunctioncall{length}(infDates) > 0) \{
        fixSummary[\hlfunctioncall{which}(\hlfunctioncall{is.infinite}(fixSummary$minVal) == TRUE), ]$minVal <- NA
        fixSummary[\hlfunctioncall{which}(\hlfunctioncall{is.infinite}(fixSummary$maxVal) == TRUE), ]$maxVal <- NA
    \}
    
\hlcomment{    # And Replace the original Values}
    \hlfunctioncall{for} (i in 1:\hlfunctioncall{nrow}(fixSummary)) \{
        thisRow <- fixSummary[i, ]
        rowIdx <- \hlfunctioncall{which}(houseSummary$nodeId == thisRow$nodeId & houseSummary$locationId == 
            thisRow$locationId & houseSummary$type == thisRow$type & houseSummary$DT == 
            thisRow$dt)
        houseSummary[rowIdx, ]$count <- thisRow$tCount
        houseSummary[rowIdx, ]$countWithBad <- thisRow$count
        houseSummary[rowIdx, ]$minVal <- thisRow$minVal
        houseSummary[rowIdx, ]$maxVal <- thisRow$maxVal
    \}
\}  \hlcomment{#End of IF Statement}
\end{alltt}


{\ttfamily\noindent\textcolor{warningcolor}{\#\# Warning: no non-missing arguments to min; returning Inf}}

{\ttfamily\noindent\textcolor{warningcolor}{\#\# Warning: no non-missing arguments to max; returning -Inf}}

{\ttfamily\noindent\textcolor{warningcolor}{\#\# Warning: no non-missing arguments to min; returning Inf}}

{\ttfamily\noindent\textcolor{warningcolor}{\#\# Warning: no non-missing arguments to max; returning -Inf}}

{\ttfamily\noindent\textcolor{warningcolor}{\#\# Warning: no non-missing arguments to min; returning Inf}}

{\ttfamily\noindent\textcolor{warningcolor}{\#\# Warning: no non-missing arguments to max; returning -Inf}}

{\ttfamily\noindent\textcolor{warningcolor}{\#\# Warning: no non-missing arguments to min; returning Inf}}

{\ttfamily\noindent\textcolor{warningcolor}{\#\# Warning: no non-missing arguments to max; returning -Inf}}

{\ttfamily\noindent\textcolor{warningcolor}{\#\# Warning: no non-missing arguments to min; returning Inf}}

{\ttfamily\noindent\textcolor{warningcolor}{\#\# Warning: no non-missing arguments to max; returning -Inf}}

{\ttfamily\noindent\textcolor{warningcolor}{\#\# Warning: no non-missing arguments to min; returning Inf}}

{\ttfamily\noindent\textcolor{warningcolor}{\#\# Warning: no non-missing arguments to max; returning -Inf}}

{\ttfamily\noindent\textcolor{warningcolor}{\#\# Warning: no non-missing arguments to min; returning Inf}}

{\ttfamily\noindent\textcolor{warningcolor}{\#\# Warning: no non-missing arguments to max; returning -Inf}}

{\ttfamily\noindent\textcolor{warningcolor}{\#\# Warning: no non-missing arguments to min; returning Inf}}

{\ttfamily\noindent\textcolor{warningcolor}{\#\# Warning: no non-missing arguments to max; returning -Inf}}\end{kframe}
\end{knitrout}


Figure~\ref{fig:remove-calib-data}, demonstrates calibration and removal of
erroneous data,  samples in \emph{blue} are removed from the database

\begin{figure}
\begin{knitrout}
\definecolor{shadecolor}{rgb}{0.969, 0.969, 0.969}\color{fgcolor}\begin{kframe}
\begin{alltt}
plt <- \hlfunctioncall{ggplot}(\hlfunctioncall{subset}(fixCalib, type == 0))
plt <- plt + \hlfunctioncall{geom_point}(\hlfunctioncall{aes}(dt, calibValue, color = badValue))
plt <- plt + \hlfunctioncall{ylab}(\hlstring{"Value"}) + \hlfunctioncall{xlab}(\hlstring{"Date"})
plt <- plt + \hlfunctioncall{opts}(title = \hlstring{"Calibrating and Removing \hlstring{'Bad'} data"})
plt <- plt + \hlfunctioncall{facet_grid}(nodeId ~ .)
\hlfunctioncall{print}(plt)
\end{alltt}
\end{kframe}
\includegraphics[width=\maxwidth]{figure/unnamed-chunk-10} 

\end{knitrout}

\caption{Calibrate and Remove ``bad'' samples}
\label{fig:remove-calib-data}
\end{figure}

\subsection{Calculating Per Node Yields}

First a Yield per Sensor is Calculated, This is based on the \emph{ideal} yield
of 288 samples per day (5 minute sampling interval)

\begin{equation}
  \text{Yield} = \text{count} * 288 / 100.0
  \label{eq:calc-yield}
\end{equation}

\begin{figure}
  \label{fig:sensor-yield}

\begin{knitrout}
\definecolor{shadecolor}{rgb}{0.969, 0.969, 0.969}\color{fgcolor}\begin{kframe}
\begin{alltt}
\hlcomment{# Work out the first and last samples so we can get yields for each node /}
\hlcomment{# Day print('FIRST AND LAST SAMPLES') First and Last Samples}
firstSample <- \hlfunctioncall{as.character}(\hlfunctioncall{min}(houseSummary$dt))
lastSample <- \hlfunctioncall{as.character}(\hlfunctioncall{max}(houseSummary$dt))
houseData[rowNo, ]$dataStart <- firstSample
houseData[rowNo, ]$dataEnd <- lastSample

\hlcomment{# Error check, if there is no start date / end date in the database, We}
\hlcomment{# use the date of the first and last sample}
hSd <- \hlfunctioncall{as.POSIXlt}(theHouse$sd, tz = \hlstring{"GMT"})
\hlfunctioncall{if} (\hlfunctioncall{is.na}(hSd)) \{
    hSd <- \hlfunctioncall{as.POSIXlt}(firstSample, tz = \hlstring{"GMT"})
\}
hEd <- \hlfunctioncall{as.POSIXlt}(theHouse$ed, tz = \hlstring{"GMT"})
\hlfunctioncall{if} (\hlfunctioncall{is.na}(hEd)) \{
    hEd <- \hlfunctioncall{as.POSIXlt}(lastSample, tz = \hlstring{"GMT"})
\}

\hlcomment{# Calculate the Yield per Node / Sensor / Day}
houseSummary$dayYield <- (houseSummary$count/288) * 100

plt <- \hlfunctioncall{ggplot}(houseSummary)
plt <- plt + \hlfunctioncall{geom_point}(\hlfunctioncall{aes}(dt, dayYield, color = type))
plt <- plt + \hlfunctioncall{geom_step}(\hlfunctioncall{aes}(dt, dayYield, color = type))
plt <- plt + \hlfunctioncall{ylab}(\hlstring{"Yield %"}) + \hlfunctioncall{xlab}(\hlstring{"Date"})
plt <- plt + \hlfunctioncall{opts}(title = \hlfunctioncall{paste}(hseName, \hlstring{" Yield per Sensor"}))
plt <- plt + \hlfunctioncall{facet_grid}(nodeId ~ .)
\hlfunctioncall{print}(plt)
\end{alltt}
\end{kframe}
\includegraphics[width=\maxwidth]{figure/unnamed-chunk-11} 

\end{knitrout}

\caption{Yield Per Sensor}
\end{figure}

Next summary yields per node / location are calculated, these summaries include
minimum, maximum and average yields per sensor.  Minimum and Maximum values are
included as in some cases, the yield for individual sensors varies on each node.
For Example, consider a case where the battery is running low (below ~2.3V), in
this case the readings from the attached Temperature and Humidity sensors will
have errors, while other sensors (such as light levels) will continue to be
reported correctly.  Therefore each sensor may have differing yields.

\begin{figure}
  \label{fig:node-yield}

\begin{knitrout}
\definecolor{shadecolor}{rgb}{0.969, 0.969, 0.969}\color{fgcolor}\begin{kframe}
\begin{alltt}
\hlcomment{# Averge out the Yields by Node / Location / Date (IE Combine all Sensors}
\hlcomment{# together)}
avgYield <- \hlfunctioncall{ddply}(houseSummary, \hlfunctioncall{.}(dt, nodeId, locationId), summarise, min = \hlfunctioncall{min}(dayYield), 
    max = \hlfunctioncall{max}(dayYield), dayYield = \hlfunctioncall{mean}(dayYield))

dayCountId = summaryData[\hlfunctioncall{which}(summaryData$name == \hlstring{"Day Count"}), ]$id
dayCountCleanId = summaryData[\hlfunctioncall{which}(summaryData$name == \hlstring{"Day \hlfunctioncall{Count} (Clean)"}), 
    ]$id
\hlcomment{# print(paste('Day Count Id ',dayCountId)) print(paste('Day Count Clean Id}
\hlcomment{# ',dayCountCleanId))}

plt <- \hlfunctioncall{ggplot}(avgYield)
plt <- plt + \hlfunctioncall{geom_point}(\hlfunctioncall{aes}(dt, dayYield))
plt <- plt + \hlfunctioncall{geom_errorbar}(\hlfunctioncall{aes}(dt, ymin = min, ymax = max))
plt <- plt + \hlfunctioncall{facet_grid}(nodeId ~ .)
\hlfunctioncall{print}(plt)
\end{alltt}
\end{kframe}
\includegraphics[width=\maxwidth]{figure/unnamed-chunk-12} 

\end{knitrout}

\caption{Yield per Node}
\end{figure}

\subsection{Calculating Deployment Yield}



\subsection{Insert Summary Information into Main Database}
Summary information is inserted into the main database.

% <<echo=FALSE>>=
% # Raw Counts
% countInsert <- data.frame(time=houseSummary$dt,
%                           nodeId=houseSummary$nodeId,
%                           sensorTypeId=houseSummary$type,
%                           summaryTypeId=NA,
%                           locationId=houseSummary$locationId,
%                           value=houseSummary$countWithBad
%                           )

% countInsert$summaryTypeId <- dayCountId
% dbWriteTable(con,"Summary",countInsert,append=TRUE,row.name=FALSE)

%  print("Counting Clean Samples")
% #Clean counts
% countInsert <- data.frame(time=houseSummary$dt,
%                           nodeId=houseSummary$nodeId,
%                           sensorTypeId=houseSummary$type,
%                           summaryTypeId=dayCountCleanId,
%                           locationId=houseSummary$locationId,
%                           value=houseSummary$count
%                           )

% dbWriteTable(con,"Summary",countInsert,append=TRUE,row.name=FALSE)
% @ 


\end{document}
