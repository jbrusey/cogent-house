\documentclass[10pt,a4paper]{article}
\usepackage[OT1]{fontenc}
\usepackage{Sweave}

\title{Deployment Data Cleanup}
\author{D Goldsmith, R Wilkins}

\begin{document}
\maketitle

\section{Introduction}

Below is a description of the data summary process, how the information is
processed an yields calculated.

It takes the form of a walk though of the yield calculation process, and
includes code snippets from the R script used to generate the data.

The document was generated in R and Latex, using the Sweave plugin.


\section{Summary Table}
A new table added to the database, the summary table is intended to hold summary
statistics on deployments.  This means that future work can avoid having to
process entire data sets when dealing with yield, or other summarised functions.  

The summary table takes the same form as the reading table, with an additional
\emph{summary type} column,  these summary types are taken from a lookup
table in the database.

Database Rows, and expected inputs are given below

\begin{table}[htbp]
  \centering
  \begin{tabular}{l l p{6cm}}
    Row & Type & Description \\ \hline
    Time & PK,Required & Timestamp of summary, In general I would expect
  this to use midnight to summarise a complete day. However, if more detailed
  summaries (such as hourly) are needed, this should not be a problem.\\
    nodeId &PK,Required & Id of node that this summary is from\\
    sensorTypeId &PK & Id of sensor that this summary is from,  this
  can be left NULL to indicate whole node summary samples (for example yield)\\
    summaryTypeId &FK &  Id of summary type.\\
    locationId & FK & Id of location this node is from,  to keep
  parity with the reading table\\
    value & float& Value of the summary\\
    textValue & string(30)& Optional text description of the summary, for example ``Hot''
  if we are dealing with exposure graphs.\\
    
  \end{tabular}
  \caption{Summary Table Description}
\end{table}

\section{Scripts}
This section has a description of the scripts used process the data, and combine
all samples into one database.

These scripts are designed to work with the new format (location aware) database format.

They can found in the \emph{dataclense} directory of the \emph{cogent-house/djgoldsmith-devel} repository.

\begin{description}
    \item[processCC.py]  Transfers current cost data from the old style sqlite
      database, into the new format database.
    \item[processAr.py] Transfers data from an Archrock postgresql database into the
      new format database.
    \item[getStats.R] R script that calculates yields for each deployment in a given database
    \item[calcKwh.R] R script to caluclate KwH usage from current cost readings.
\end{description}

Further details of these scripts are given below

\section{getStats.R}

This script calculates summary statistics for all houses in a given database.
The statistics are output in two formats.
\begin{itemize}
  \item \emph{.csv} file with summary output for this database
  \item update rows in the \emph{summary} table given these statistics
\end{itemize}

To run the script modify the source file with the relevant database access
name. Then run the script through R. 


\subsection{Script initialisation}

\begin{itemize}
\item Load the relevant R librarys
\item Connects to the database
\item Loads the Relevant Tables

\end{itemize}


\end{document}
